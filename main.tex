%----------------------------------------------------------------------
%	Template de Informes v2.0
%	Autor: José Guillermo Ponce Ramírez, a partir de Templates
%	varios, y mucho Google.
%----------------------------------------------------------------------

%----------------------------------------------------------------------
%	Configuración del documento
%----------------------------------------------------------------------
\documentclass[12pt, letterpaper]{article} 
% report: Estilo Informe completo, con capítulos.
% article: Estilo Informe básico, sin Capítulos.

\usepackage[utf8x]{inputenc}
\usepackage[activeacute,spanish,es-tabla]{babel}
\usepackage[left=2cm, right=2cm, bottom=3cm, top=3cm, headheight=40pt]{geometry}
\usepackage{graphicx} % Required for including pictures
\usepackage{float} % Allows putting an [H] in \begin{figure} to specify the exact location of the figure
\usepackage{wrapfig} % Allows in-line images
\usepackage[nottoc, notlot, notlof]{tocbibind} % Índices, sin ToC, LoF ni LoT
\renewcommand{\refname}{Bibliografía} % Nombre para Bibliografía (clase article)
%\renewcommand{\bibname}{Bibliografía} % Nombre para Bibliografía (clase book/report)
\renewcommand\tocbibname{Bibliografía}
\usepackage{amsmath}
\usepackage{amsfonts}
\usepackage{amssymb}
\usepackage{siunitx} % Unidades del SI
\sisetup{output-decimal-marker = {,}}
\usepackage{cancel} % Permite cancelar (tachar) elementos
\usepackage{tabu} % Tablas chéveres
\usepackage{booktabs} % Allows the use of \toprule, \midrule and \bottomrule in tables for horizontal lines
\usepackage{multirow} % Celdas en más de una fila
\usepackage{easybmat} % Matrices por Bloques
\usepackage{lipsum} % Lorem Ipsum

\title{Template Informes 2.0} % Nombre archivo en Overleaf
\newcommand{\titulo}{Título del informe}
\newcommand{\ramo}{XXNNNN Ramo}
\newcommand{\departamento}{Departamento de Ingeniería XXXXX}

\usepackage{color}
\definecolor{gray51}{rgb}{0.51,0.51,0.51}
\definecolor{gray71}{rgb}{0.71,0.71,0.71}
\newcommand{\HRule}{\rule{\linewidth}{.4mm}}

\usepackage{listings} % Incluye códigos
\renewcommand{\lstlistingname}{Algoritmo}
\renewcommand{\lstlistlistingname}{Índice de \lstlistingname s}
\lstset{ 
	basicstyle=\ttfamily\small,
    commentstyle=\color{red},
    keywordstyle=\color{blue},
    numberstyle=\tiny\color{gray71},
    stringstyle=\color{green},
    breakatwhitespace=false,         
    breaklines=true,                 
    captionpos=b,                    
    keepspaces=true,                 
    numbers=left,                    
    numbersep=5pt,                  
    showspaces=false,                
    showstringspaces=false,
    showtabs=false,                  
    tabsize=2,
    xleftmargin=2em,
    frame=single,
    framexleftmargin=1.5em
}

\usepackage{hyperref} 		% Hipervínculos
\hypersetup{
	colorlinks	= true,		% Vínculos coloreados en vez de recuadros
    urlcolor	= blue,		% Color para vínculos externos
    linkcolor	= black,	% Color para vínculos internos
    citecolor	= red		% Color para citas
}

\usepackage{fancyhdr}
\pagestyle{fancy}
\fancyhead[L]{\footnotesize Universidad de Chile - Facultad de Ciencias Físicas y Matemáticas\\
\departamento\\
\ramo\ - Semestre Otoño 2017}
\fancyhead[R]{\includegraphics[scale=0.2]{fcfm.png}}
\fancyfoot[L]{\small \rm \textit{\titulo}}
\fancyfoot[C]{}
\fancyfoot[R]{\small \rm \textbf{\thepage}}
\renewcommand{\headrulewidth}{0.5pt}
\renewcommand{\footrulewidth}{0.5pt}

%\linespread{1.2} 				% Interlineado
%\setlength\parindent{0pt} 		% Longitud sangría

\begin{document}
%----------------------------------------------------------------------
%	Portada
%----------------------------------------------------------------------
\newgeometry{left=2.5cm,right=2.5cm, top=2.5cm, bottom=2.5cm}

\begin{titlepage}
{
\begin{wrapfigure}{l}{1cm}
\vspace{-0.7cm}
\includegraphics[width=5cm]{fcfm.png}
\end{wrapfigure}

\textsc{\color{red}\hspace{3.2cm}\departamento}\\
\textsc{\color{gray51}\hspace{3.8cm}Facultad de Ciencias Físicas y Matemáticas}\\
\textsc{\color{gray51}\hspace{3.8cm}Universidad de Chile}\\
\textsc{\color{gray51}\hspace{3.8cm}\ramo}\\
}

\begin{center}
~\\[5cm]
{\color{gray71}\textsc{HiperTítulo}}
\HRule~ \\[0.4cm]
{ \Huge \textup \bfseries  \titulo}\\[0.4cm]
{ \Large \textup{Generación, parámetros de operación y estimación de instalación según demanda}}\\[0.2cm]
\HRule 
~\\[3.5cm]
\end{center}

\begin{minipage}{.5\textwidth}
~
\end{minipage}
\begin{minipage}{.45\textwidth}
\begin{flushright}
\begin{tabular}{l}
\textbf{\textit{Profesores:}} \\
{\small Claudia Rahmann Z.}\\
{\small Luis Vargas D.}\\[0.3cm]
\textbf{\textit{Auxiliar:}} \\
{\small Nicolás Mira G.}\\[0.3cm]
%\textbf{\textit{Ayudantes de}} \\
%\textbf{\textit{Laboratorio:}} \\
\textbf{\textit{Integrantes:}}\\
{\small Ignacio Ñancupil Q.}\\
{\small José Ponce R.}\\
{\small Javier Rojas J.}\\[.3cm]
\textbf{\textit{Fecha:}}\\
{\small DD de MMMM de AAAA}
\end{tabular}
\end{flushright}
\end{minipage}
\begin{minipage}{.05\textwidth}
~
\end{minipage}
\end{titlepage}

\restoregeometry
%----------------------------------------------------------------------
%	Documento
%----------------------------------------------------------------------

\pagenumbering{Roman}
\setcounter{page}{1}
\tableofcontents 
\newpage
\listoffigures
\listoftables
\lstlistoflistings

%\addcontentsline{toc}{chapter}{Nombre Sección} % Para forzar aparición en el Índice

\newpage
\pagenumbering{arabic}
\setcounter{page}{1}

\newpage
\section{Introducción}

\begin{wrapfigure}[9]{L}{0.43\textwidth}
\vspace{-7mm}
\begin{equation*}
\mathcal{C} = \left[ \begin{BMAT}(c)[4pt]{cc:cc:cc:c}{cc:cc:cc:c}
%(c: columnas del mismo ancho)[ancho columnas]{patrón columnas}{patrón filas}
1 & -1& 0 & & & & 0 \\
-1&  1& 0 & 0 & 0 & & \\
0 & 0 & \bullet & \bullet & 0 & & \\
  & 0 & \bullet & \bullet & 0 & 0 & 0\\
  &   & 0 & 0 & \bullet & \bullet & 0 \\
  &   &   & 0 & \bullet & \bullet & 0 \\
0 &   &   & 0 & 0 & 0 & \smash{\ddots}
\end{BMAT} \right]
\end{equation*}
\end{wrapfigure}

\lipsum[2-4] %[párrafo inicial - párrafo final] Lorem Ipsum, máx 150.

\newpage
\section{Marco Teórico}

\begin{equation}
\left[
  \begin{BMAT}[5pt]{c|c}{c}
    \begin{BMAT}[5pt]{c:c:c}{ccccccc}
      & & \\
      & & \\
      & & \\
      v_1 & \dots & v_m \\
      & & \\
      & & \\
      & &
    \end{BMAT}
  &
    \begin{BMAT}{c}{c|c}
      \begin{BMAT}[5pt]{c:c:c}{ccc}
        & & \\
        w_1(a) & \dots & w_n(a) \\
        & &
      \end{BMAT}
    \\
      \begin{BMAT}[10pt]{c}{c}
        B(z)
      \end{BMAT}
    \end{BMAT}
  \end{BMAT} 
\right]
\end{equation}

Matriz por bloques, con el comando BMAT

\lstinputlisting[language=Verilog, caption=Ejemplo lenguaje Verilog]{counter_verilog.v}

\newpage
\section{Resultados}

\subsection{Actividad 1: Curva Característica del Panel}
En esta primera actividad se busca obtener la curva que indica las características del panel utilizado. Los datos de placa del panel se muestran en la figurassss
%\begin{figure}[H]
%\centering
%\includegraphics[width = 0.5\textwidth]{placa.jpg}
%\caption{Datos de Placa del Panel utilizado}
%\label{v1}
%\end{figure}
Se ubica el panel de forma perpendicular al sol y se proceden a medir las corrientes de corto circuito, $I_{sc}$, tensión de circuito abierto, $V_{oc}$ y medidas de tensión y corriente para distintas cargas. Los resultados obtenidos se representan en el gráfico de la figura . \cite{explorador}




\subsubsection{Tensión e intensidad de Operación}

\begin{table}[H]
\centering
\begin{tabu} to \textwidth {l X[c] X[c] X[c]}
\toprule
\textbf{Mes} & \textbf{Energía [MWh]} & \textbf{Ingresos Mensuales [US\$]} & \textbf{Ingresos Acumulados [US\$]}\\\midrule
Enero        & 0,510 & 40,81 & 40,81  \\
Febrero      & 0,452 & 36,16 & 76,98  \\
Marzo        & 0,364 & 29,12 & 106,10 \\
Abril        & 0,377 & 30,16 & 136,26 \\
Mayo         & 0,476 & 38,10 & 174,36 \\
Junio        & 0,592 & 47,42 & 221,78 \\
Julio        & 0,545 & 43,61 & 265,39 \\
Agosto       & 0,598 & 47,90 & 313,30 \\
Septiembre   & 0,399 & 31,97 & 345,27 \\
Octubre      & 0,415 & 33,25 & 378,53 \\
Noviembre    & 0,460 & 36,87 & 415,40 \\
Diciembre    & 0,540 & 43,27 & 458,68 \\
\bottomrule
\end{tabu}
\caption{Estimación de ingresos mensual estimado a partir de la simulación.}
\label{tabla2}
\end{table}

\begin{table}[H]
\centering
\begin{tabu} to \textwidth {l X[c] X[c] X[c] X[c]}
\toprule
\multirow{2}{*}{\textbf{Carga Conectada}} & \multicolumn{2}{c}{Panel acostado} & \multicolumn{2}{c}{Panel apuntando al Este}\\\cmidrule{2-5}
& \textbf{Voltaje [V]} & \textbf{Corriente [A]} & \textbf{Voltaje [V]} & \textbf{Corriente [A]} \\\midrule
Circuito abierto 		& 38,7 	& 0 	& 38,3 	& 0 \\
1 par de ampolletas 	& 17,4 	& 1,5 	& 8		& 1 \\
2 pares de ampolletas 	& 3 	& 1,5 	& 1,4	& 1,03 \\
Cortocircuito 			& 0 	& 1,4 	& 0		& 1,9 \\
\bottomrule
\end{tabu}
\caption{Resultados obtenidos para la Actividad 2}
\label{resultadosact2}
\end{table}

\newpage
\begin{thebibliography}{99}
\bibitem{explorador} Departamento de Geofísica y Ministerio de Energía, \textit{Explorador Eólico} [online]. Disponible en \url{http://walker.dgf.uchile.cl/Explorador/Eolico2/}.

\bibitem{apunte} L. Vargas, J. Haas, F. Barría y L. Reyes, ``Energía Eólica'' en \textit{Generación de Energía Eléctrica con Fuentes Renovables} [online]. Primavera 2010, pp. 97-133. Disponible en \url{https://www.u-cursos.cl/ingenieria/2017/1/EL6000/1/material_docente/bajar?id_material=1683799}.

\bibitem{Bergey} Ficha Técnica Aerogenerador Bergey XL. 1 [online], Bergey Windpower. Disponible en \url{http://beta.energiainnovadora.com/wp-content/uploads/2017/01/Bergey-BWC-XL-1KW.pdf}.

\bibitem{simuladorConsumo} \textit{Simulador de consumo} [online]. Enel Distribución. Disponible en \url{https://www.eneldistribucion.cl/simulador-consumo}.

\bibitem{SAM} \textit{System Advisor Model (SAM)} [online]. Natoinal Renewable Energy Laboratory (NREL). Disponible en: \url{https://sam.nrel.gov/}

\bibitem{parametros} J. Mur Amada, ``Selección de Emplazamientos'' en \textit{Curso de Energía Eólica} [online], Departamento de Ingeniería Eléctrica de la Universidad de Zaragoza, pp. 32-34. Disponible en \url{http://www.windygrid.org/manualEolico.pdf}.

\end{thebibliography}
\end{document}